\subsection{B2B Sales}

Business-to-business (B2B) sales processes are significantly more complex compared to business-to-consumer (B2C) contexts, characterized by slower decision-making cycles and intricate information exchanges \cite{rodriguez2020digitalization}. The information requirements in B2B sales are resource-intensive, with substantial reliance on the capabilities and expertise of the sales force \cite{agnihotri2012bringing}. Unlike B2C transactions, B2B sales primarily revolve around establishing robust, long-term relationships optimally achieved through trust development \cite{rodriguez2020digitalization}.

In B2B contexts, buying firms require comprehensive engagement from selling entities, necessitating thorough information exchange that goes beyond mere product specifications \cite{forsell2024competitive}. Traditional face-to-face meetings between sellers and buyers involve both objective information and nonverbal cues, which play a critical role in establishing mutual understanding, agreement levels, and trust in the salesperson \cite{rodriguez2020digitalization, morgan1994commitment}. These interactions facilitate the sharing of both rational and emotional information, establishing a common ground that infuses the exchanged information with value and significance for both parties \cite{rodriguez2020digitalization}.

The sales argument is a meticulously crafted document highlighting product-specific selling points to better address customer needs and concerns \cite{kotler2016marketing}. It serves as a guide to structuring communication that influences the customer's decision-making process, containing various arguments designed to provide the sales force with precise knowledge of the product's features and qualities \cite{elhissoufi2024leveraging}. Effective sales argumentation cannot be improvised; it requires thorough preparation and adaptation to meet the needs and expectations of each customer rather than using standardized pitches \cite{futrell2011fundamentals}.

Among various sales techniques, the Feature-Advantage-Benefit (FAB) method is one of the most effective and practical approaches for sales argument creation and preparation \cite{elhissoufi2024leveraging}. This method clearly communicates a product's features (technical characteristics), advantages (improvements the feature brings), and benefits (personal or emotional gains from using the product) to potential buyers, linking them to consumer needs \cite{futrell2011fundamentals}. For instance, in presenting a smartphone, a salesperson would describe the high-resolution screen as a feature, improved visual clarity as its advantage, and ease of use with eye protection as the resulting benefit. This structured approach enhances the persuasiveness of the sales argument by focusing on how the product satisfies specific customer needs rather than merely describing its technical specifications \cite{elhissoufi2024leveraging}.

The integration of advanced technologies like artificial intelligence (AI) has begun transforming traditional B2B sales processes. AI technologies have been redefining sales practices, changing all sales funnel steps from making initial contact with prospects to managing negotiations \cite{paschen2020collaborative}. Generative AI, in particular, promises to disrupt B2B customer experience, productivity, and growth by enabling hyper-personalized content and offerings based on individual profiles, customer behavior, and purchase history \cite{deveau2023ai}. However, there remain challenges in aligning AI solutions with corporate strategies, with ongoing debates about their actual value in business applications \cite{forsell2024competitive}. Research indicates strategic challenges associated with AI, such as its limitation in recognizing business interdependencies and resistance to managerial control \cite{kemp2023competitive}. Despite these challenges, the potential of AI to enhance B2B sales processes through automation, personalization, and data-driven insights presents significant opportunities for competitive advantage in increasingly complex sales environments.