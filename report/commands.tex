% BibTeX definition
\def\BibTeX{{\rm B\kern-.05em{\sc i\kern-.025em b}\kern-.08em\TeX}}

% Custom title page command
\newcommand{\maketitlepage}{
    \begin{titlepage}
        \centering
        {\huge\bfseries INTERIM REPORT\\}
        \vspace*{2cm}
        {\huge\bfseries Digital Ising Machines for Postal Delivery Optimization}
        \vfill
        {\Large\itshape Thomas Pluck\\ MSc Robotics and Embedded AI Candidate \thanks{Corresponding author: thomas.pluck.2025@mumail.ie}\par}
        \vspace{1cm}
        {\large Department of Electronic Engineering\par}
        {\large Maynooth University\par}
        {\large Maynooth, Ireland\par}
        \vspace{2cm}
        
        \input{figures/mobius_logo}
        %\includegraphics[width=\textwidth]{img/rpp_instance.png}
        
        \vfill
    \end{titlepage}
}

% Mathematical notation shortcuts
\newcommand{\R}{\mathbb{R}}        % Real numbers
\newcommand{\Z}{\mathbb{Z}}        % Integers
\newcommand{\N}{\mathbb{N}}        % Natural numbers
\newcommand{\T}{\mathbb{T}}        % Torus
\newcommand{\C}{\mathbb{C}}        % Complex numbers

% Vector notation
\newcommand{\vect}[1]{\mathbf{#1}} % Bold vector notation
\newcommand{\mat}[1]{\mathbf{#1}}  % Bold matrix notation

% Probability and statistics
\newcommand{\E}{\mathbb{E}}        % Expected value
\newcommand{\Var}{\mathrm{Var}}    % Variance
\newcommand{\Cov}{\mathrm{Cov}}    % Covariance
\newcommand{\Prob}{\mathbb{P}}     % Probability

% Physics notation
\newcommand{\dd}{\mathrm{d}}       % Differential operator
\newcommand{\pdiff}[2]{\frac{\partial #1}{\partial #2}} % Partial derivative

% Custom environments
\newenvironment{theorem}[1][\unskip]{\begin{trivlist}\item[\hskip \labelsep {\bfseries Theorem #1.}]}{\end{trivlist}}
\newenvironment{lemma}[1][\unskip]{\begin{trivlist}\item[\hskip \labelsep {\bfseries Lemma #1.}]}{\end{trivlist}}
\newenvironment{proof}[1][\unskip]{\begin{trivlist}\item[\hskip \labelsep {\itshape Proof #1.}]}{\hfill$\square$\end{trivlist}}

% Circuit diagram shortcuts
\newcommand{\buffer}{\circled{>}}   % Buffer symbol
\newcommand{\inverter}{\circled{1}} % Inverter symbol
\newcommand{\coupling}{\text{---}}  % Coupling line

% Section referencing
\newcommand{\secref}[1]{Section~\ref{#1}}
\newcommand{\figref}[1]{Figure~\ref{#1}}
\newcommand{\tabref}[1]{Table~\ref{#1}}

% Algorithm notation
\newcommand{\BigO}[1]{\mathcal{O}(#1)}  % Big-O notation
\newcommand{\complexity}[1]{\text{#1}}   % Complexity class

% Quantum mechanics notation (since you discuss quantum annealing)
\newcommand{\bra}[1]{\left\langle#1\right|}
\newcommand{\ket}[1]{\left|#1\right\rangle}
\newcommand{\braket}[2]{\left\langle#1|#2\right\rangle}

% Complexity notation
\newcommand{\Poly}{\mathcal{P}}
\newcommand{\NP}{\mathcal{NP}}