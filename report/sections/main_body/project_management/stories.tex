\subsection{User Stories}

User stories represent concise, user-centric descriptions of software functionality valuable to users or customers of a system \cite{cohn2004user}\cite{cohn2004advantages}. Commonly employed in Agile development methodologies such as Scrum and Extreme Programming (XP), user stories serve as a means to clearly articulate user needs and facilitate project planning \cite{cohn2004user}\cite{cohn2004advantages}\cite{cao2008agile}. In this project, each user story comprises three essential elements: a written description, stakeholder conversations to detail the requirements, and acceptance tests to verify completion \cite{cohn2004user}\cite{cao2008agile}\cite{lucassen2016improving}.

User stories are widely adopted due to their simplicity, flexibility, and effectiveness in promoting clear communication among stakeholders \cite{choma2016userx}\cite{cohn2004advantages}. Unlike more extensive documentation methods, user stories are concise and written in accessible language, promoting broader participation from the entire Agile team, including end-users \cite{choma2016userx}. This participatory approach ensures the product backlog accurately reflects user needs, goals, and evolving requirements throughout the project lifecycle \cite{choma2016userx}\cite{cao2008agile}.

However, the simplicity of user stories can lead to ambiguity, resulting in vague, unclear, or imprecise interpretations \cite{amna2022ambiguity}. Such ambiguity may create challenges in clearly articulating choices and expectations, thereby impacting development clarity and precision \cite{amna2022ambiguity}. Addressing this ambiguity is essential for effective requirement communication and alignment among stakeholders.

To enhance clarity and quality, frameworks such as the Quality User Story (QUS) have been proposed, outlining criteria like unambiguity, atomicity, minimalism, and completeness \cite{lucassen2016improving}. Following these quality criteria ensures user stories are well-defined within this project, estimable, independent, and problem-oriented, significantly enhancing their usefulness in Agile planning and execution \cite{lucassen2016improving}.

Usability is another critical factor addressed by user stories, emphasizing the need for the system to provide intuitive and satisfactory feedback to users \cite{moreno2012agile}. Usability-specific user stories, termed usability stories, explicitly document the features required to improve user trust, ease of use, and overall user experience \cite{moreno2012agile}. Incorporating usability considerations into user stories helps ensure the developed system aligns closely with end-user expectations and requirements \cite{moreno2012agile}.

Incorporating user stories into Scrum practices provides agility, allowing teams to respond effectively to rapidly changing requirements typical in software development \cite{cao2008agile}. Through iterative planning and continuous stakeholder feedback, user stories remain relevant and accurately represent current user needs, helping to manage evolving requirements proactively \cite{cao2008agile}. This project synthesizes user stories as an integral part of the Scrum methodology, leveraging their strengths to facilitate continuous communication, prioritization, and refinement of requirements, ultimately delivering valuable and user-driven outcomes.
