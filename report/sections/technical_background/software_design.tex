\subsection{Software Design}

Software design is a fundamental aspect of application development, encompassing the systematic planning and structuring of software systems to achieve specific functional and performance goals \cite{yau1986survey}. Effective software design prioritizes modularity, maintainability, scalability, and adaptability. These attributes facilitate the management of complexity inherent in software projects, supporting the incremental and iterative nature of software development, and enabling flexibility in response to evolving requirements \cite{tang2010software}. This is especially important in B2B sales where application requirements can grow or change rapidly \cite{rodriguez2020digitalization}.

One widely recognized approach in software design is the concept of separation of concerns, a principle that advocates dividing software into distinct, clearly defined components or modules \cite{moreira2005multi}. Each component is responsible for specific functionalities and can be developed and tested independently \cite{yau1986survey}. This modularity simplifies the integration and testing processes, reducing complexity and improving maintainability \cite{song1994engineering}. Clearly defined interfaces and responsibilities among software components significantly enhance both collaborative development efforts and long-term software quality \cite{yau1986survey}.

Requirement analysis is another critical step in software design, serving as the foundation for setting clear design goals \cite{sonnentag2006expertise}. Thorough analysis involves systematically identifying, documenting, and validating user and system requirements, ensuring alignment between stakeholder expectations and software capabilities \cite{yau1986survey}. Effective requirement analysis not only captures explicitly stated customer needs but also includes inferring implicit requirements, uncovering user expectations that may not be directly communicated \cite{tang2010software}. Such proactive inference ensures comprehensive design coverage, contributing significantly to system usability and customer satisfaction.

A crucial element highlighted frequently in software design literature is robust error handling. Robust error handling involves proactive strategies to anticipate, detect, manage, and recover from errors to ensure system reliability and user confidence \cite{yau1986survey}. It is recommended to incorporate explicit error checking, recovery procedures, and user communication strategies to mitigate adverse effects and enhance user experience. Additionally, effective software design for AI-based programs must include robust monitoring and logging mechanisms to ensure reliability and predictability in operation \cite{bosch2020engineering}. These measures support system stability and facilitate quick troubleshooting, which are critical for maintaining software \cite{yau1986survey}.

Recent literature provides valuable insights into software design patterns specifically for AI-based systems. A multi-vocal literature review has identified various established and emerging patterns applicable to AI systems, such as the Multi-Layer Pattern (also known as Separation of Concerns or Multi-Tiered Architecture) \cite{heiland2023design}. This pattern divides applications into clearly defined layers, each consisting of submodules that process input and provide output to subsequent layers. Such an approach allows for independent layer design and enhances the adaptability of AI systems. The modular structure facilitates the inference of results at each layer, enabling easier adjustments and scalability, particularly beneficial in complex applications such as those involving AI components. 
